% -*- coding: utf-8 -*-
% !TEX program = xelatex
\documentclass{jnuexam}

%\answerfalse %不显示答案

\setlength\arraycolsep{4pt}
\newcommand{\cov}{\operatorname{cov}}

\begin{document}

\renewcommand{\niandu}{2018--2019}
\renewcommand{\xueqi}{1}
\renewcommand{\kecheng}{合同法}
\renewcommand{\zhuanye}{法学} % 可以为空白
\renewcommand{\jiaoshi}{XXX} % 教师姓名
\renewcommand{\shijian}{2019~年~01~月~28~日}
\renewcommand{\bixiu}{1} % 1 为必修,0 为选修
\renewcommand{\bijuan}{1} % 1 为闭卷,0 为开卷
\renewcommand{\shijuan}{A} % A 或 B 或 C 卷
\renewcommand{\neizhao}{1} % 1 打勾,0 不勾
\renewcommand{\waizhao}{0} % 1 打勾,0 不勾

\makehead % 生成试卷表头

\makepart{填空题}{共~8~小题,每小题~2~分,共~16~分}

\newpageb % B卷分页点

\begin{problem}
已知二阶行列式 $\text{$\left|\begin{array}{cc}
  1 & 2\\
  - 3 & x
\end{array}\right|$=0}$,则 $x=$ \fillout{$-6$}。
\end{problem}

\vfill

\begin{problem}
五阶行列式的一共有 \fillout{$120$} 项。
\end{problem}

\vfill

\begin{problem}
向量组 $\alpha_1=(1,1,0), \alpha_2=(0,1,1), \alpha_3=(1,0,1)$,
则将向量 $\beta=(4, 5, 3)$ 表示为 $\alpha_1, \alpha_2, \alpha_3$
的线性组合为 $\beta=$ \fillout{$3\alpha_1+2\alpha_2+\alpha_3$}。
\end{problem}

\vfill

\begin{problem}
已知$P(A)=0.3$, $P(B|A)=0.4$, $P(B|\bar{A})=0.5$, 则$P(B)=$ \fillout{$0.47$}。
\end{problem}

\vfill

\begin{problem}
已知连续型$\xi$的密度函数为$\varphi(x)=\left\{
\begin{array}{ll}
  k \cos x, & - \frac{\pi}{2} < x < \frac{\pi}{2}\\
  0, & \text{其它}
\end{array}\right.$,
则$k=$ \fillout{$\frac{1}{2}$}。
\end{problem}

\vfill

\begin{problem}
已知随机变量$\xi$的期望和方差各为$E\xi=3, D\xi=2$, 则$E\xi^2=$ \fillout{$11$}。
\end{problem}

\vfill

\begin{problem}
电子管寿命$\xi$满足平均寿命为$1000$小时的指数分布,则它的寿命小于$2000$小时概率为 \fillout{$1-e^{-2}$}。
\end{problem}

\vfill

\begin{problem}
已知$\xi$和$\eta$相互独立且$\xi\sim N(1,4), \eta\sim N(2,5)$,则$\xi-2\eta\sim$ \fillout{$N(-3,24)$}。
\end{problem}

\vfill

\newpagea % A卷分页点

\makepart{单选题}{共~20~小题,每小题~2~分,共~40~分}

\newpageb % B卷分页点

\begin{problem}
我国民法的调整对象是 \pickout{C}
\fullitem{一切横向经济关系}
\fullitem{法定范围的财产关系和人身关系}
\fullitem{平等主体间的财产关系和人身关系}
\fullitem{平等主体间的人身关系和完全没有国家参与的财产关系}
\end{problem}

\vfill

\begin{problem}
某媒体未征得艾滋病孤儿小明的同意,发表了一首关于小明的报道,将其真实姓名、照片和患病经历公之于众。报道发表后,隐去真实身份开始正常生活的小明再次受到歧视和排斥,下列哪选项是正确的? \pickout{A}
\halfitem{该媒体的行为不构成侵权}
\halfitem{该媒体侵犯了小明的健康权}
\halfitem{该媒体侵犯了小明的姓名权}
\halfitem{该媒体侵犯了小明的隐私权}
\end{problem}

\vfill

\begin{problem}
已知矩阵 $A = \left(\begin{array}{ccc}
  1 & 1 & 0\\
  1 & x & 0\\
  0 & 0 & 1
\end{array}\right)$ 其中两个特征值为 $\lambda_1 = 1$ 和 $\lambda_2
= 2$,则 $x=$ \pickout{B}
\halfitem{该媒体的行为不构成侵权}
\halfitem{$1$}
\halfitem{$1$}
\halfitem{$1$}
\end{problem}

\vfill

\begin{problem}
二次型 $f = 4 x_1^2 - 2 x_1 x_2 + 6 x_2^2$ 对应的矩阵等于 \pickout{C}
\quaritem{$\left(\begin{array}{cc}
  4 & - 2\\
  - 2 & 6
\end{array}\right)$}
\quaritem{$\left(\begin{array}{cc}
  2 & - 2\\
  - 2 & 3
\end{array}\right)$}
\quaritem{$\left(\begin{array}{cc}
  4 & - 1\\
  - 1 & 6
\end{array}\right)$}
\quaritem{$\left(\begin{array}{cc}
  2 & - 1\\
  - 1 & 3
\end{array}\right)$}
\end{problem}

\vfill

\begin{problem}
对任何一个本校男学生,以$A$表示他是大一学生,$B$表示他是大二学生,则事件$A$和$B$是\pickout{B}
\halfitem{对立事件}
\halfitem{互斥事件}
\halfitem{既是对立事件又是互斥事件}
\halfitem{不是对立事件也不是互斥事件}
\end{problem}

\vfill

\begin{problem}
下列说法\CJKunderline{不正确}的是\pickout{B}
\fullitem{大数定律说明了大量相互独立且同分布的随机变量的均值的稳定性}
\fullitem{大数定律说明大量相互独立且同分布的随机变量的均值近似于正态分布}
\fullitem{中心极限定理说明了大量相互独立且同分布的随机变量的和的稳定性}
\fullitem{中心极限定理说明大量相互独立且同分布的随机变量的和近似于正态分布}
\end{problem}

\vfill

\begin{problem}
在数理统计中,对总体$X$和样本$(X_1,\cdots,X_n)$的说法哪个是\CJKunderline{不正确}的\pickout{D}
\halfitem{总体是随机变量}
\halfitem{样本是$n$元随机变量}
\halfitem{$X_1, \cdots, X_n$相互独立}
\halfitem{$X_1 = X_2 =\cdots = X_n$}
\end{problem}

\vfill

\begin{problem}
样本平均数$\bar{X}$\CJKunderline{未必是}总体期望值$\mu$的\pickout{A}
\quaritem{最大似然估计}
\quaritem{有效估计}
\quaritem{一致估计}
\quaritem{无偏估计}
\end{problem}

\vfill

\newpagea % A卷分页点

\makepart{简答题}{共~2~小题,25~分}

\newpageb % B卷分页点

\begin{problem}
简述抗同权的内涵,特征及其类型(12分)
\end{problem}

\bigskip

\begin{solution}
$A = \left|\begin{array}{cccc}
    0 & 1 & 2 & 3\\
    1 & 2 & 3 & 0\\
    2 & 3 & 0 & 1\\
    3 & 0 & 1 & 2
  \end{array}\right| = \left|\begin{array}{cccc}
    0 & 1 & 2 & 3\\
    1 & 2 & 3 & 0\\
    0 & - 1 & - 6 & 1\\
    0 & - 6 & - 8 & 2
  \end{array}\right| = 1 \cdot (- 1)^{2 + 1} \left|\begin{array}{ccc}
    1 & 2 & 3\\
    - 1 & - 6 & 1\\
    - 6 & - 8 & 2
  \end{array}\right|$ \dotfill 4分\par
\qquad\qquad $= -\left|\begin{array}{ccc}
    1 & 2 & 3\\
    0 & - 4 & 4\\
    0 & 4 & 20
  \end{array}\right| = - \left|\begin{array}{cc}
    - 4 & 4\\
    4 & 20
  \end{array}\right| = -(-4\cdot20-4\cdot4) = 96$ \dotfill 8分
\end{solution}

\vfill

\begin{problem}
简述《民法总则)的特色亮点(13分)
\end{problem}

\bigskip

\begin{solution}
好
非常好
\end{solution}

\vfill


\newpage % A,B卷共同分页点

\begin{problem}
设每发炮弹命中飞机的概率是0.2且相互独立,现在发射100发炮弹。\par
(1) 用切贝谢夫不等式估计命中数目$\xi$在10发到30发之间的概率。\par
(2) 用中心极限定理估计命中数目$\xi$在10发到30发之间的概率。
\end{problem}

\bigskip

\begin{solution}
$E\xi = n p = 100 \cdot 0.2 = 20, D\xi = n p q = 100 \cdot 0.2 \cdot 0.8 = 16$. \dotfill 2分 \par
(1) $P (10 < \xi < 30) = P (| \xi - E \xi | < 10) \geq 1 - \frac{D\xi}{10^2}
     = 1 - \frac{16}{100} = 0.84$. \dotfill 4分 \par
(2) $P (10 < \xi < 30) \approx \Phi_0 \left( \frac{30 - 20}{\sqrt{16}}\right)
     - \Phi_0 \left( \frac{10 - 20}{\sqrt{16}} \right)$ \dotfill 6分\par
\qquad $= 2 \Phi_0 (2.5) - 1 = 2 \cdot 0.9938 - 1 =0.9876$ \dotfill 8分
\end{solution}

\vfill

\begin{problem}
从正态总体$N(\mu,\sigma^2)$中抽出样本容量为16的样本,算得其平均数为3160,标准差为100。
试检验假设$H_0:\mu=3140$是否成立($\alpha = 0.01$)。
\end{problem}

\bigskip

\begin{solution}
(1) 待检假设 $H_0 : \mu = 3140$. \dotfill 1分\par
(2) 选取统计量 $T = \frac{\bar{X}-\mu}{S / \sqrt{n}} \sim t(n-1)$. \dotfill 3分 \par
(3) 查表得到 $t_{\alpha} = t_{\alpha} (n - 1) = t_{0.01} (15) =2.947$. \dotfill 5分 \par
(4) 计算统计值 $t = \frac{\bar{x} - \mu_0}{s/\sqrt{n}} =\frac{3160-3140}{100/4} = 0.8$.\dotfill 7分 \par
(5) 由于 $| t | < t_{\alpha}$, 故接受 $H_0$, 即假设成立. \dotfill 8分
\end{solution}

\vfill

\newpagea % A卷分页点

\makepart{证明题}{共~2~小题,每小题~10~分,共~20~分}

\begin{problem}
不使用矩阵可相似对角化的判别定理,直接用矩阵的运算和性质证明下面的矩阵$A
=\left(\begin{array}{cc}
  1 & 1\\
  0 & 1
\end{array}\right)$不能相似对角化,即不存在可逆矩阵$P$和对角阵$\Lambda$使得$P^{-1}AP=\Lambda$。
\end{problem}

\bigskip

\begin{proof}
假设有$P = \left(\begin{array}{cc}
  a & b\\
  c & d
\end{array}\right)$使得$P^{-1}AP = \Lambda$,即$AP=P\Lambda$。\dotfill 2分\par
则有 $$\left(\begin{array}{cc}
  a + c & b + d\\
  c & d
\end{array}\right) = \left(\begin{array}{cc}
  1 & 1\\
  0 & 1
\end{array}\right) \left(\begin{array}{cc}
  a & b\\
  c & d
\end{array}\right) = \left(\begin{array}{cc}
  a & b\\
  c & d
\end{array}\right) \left(\begin{array}{cc}
  \lambda_1 & \\
  & \lambda_2
\end{array}\right) = \left(\begin{array}{cc}
  a \lambda_1 & b \lambda_2\\
  c \lambda_1 & d \lambda_2
\end{array}\right)$$ 因此有 $\left\{ \begin{array}{llll}
  a + c & = & a \lambda_1 & (1)\\
  b + d & = & b \lambda_2 & (2)\\
  c & = & c \lambda_1 & (3)\\
  d & = & d \lambda_2 & (4)
\end{array} \right.$ \dotfill 6分\par
由第1个和第3个方程消去$\lambda_1$,可以得到 $c^2 = 0$ 即 $c=0$;
由第2个和第4个方程消去$\lambda_2$,可以得到 $d^2 = 0$ 即 $d=0$。
因此矩阵$P$不可逆,矛盾。\dotfill 10分
\end{proof}

\vfill

\begin{problem}
设事件$A$和$B$相互独立,证明$A$和$\bar{B}$相互独立。
\end{problem}

\bigskip

\begin{proof}
$P (A \cdot \bar{B}) = P (A - B) = P (A - A B)$ \dotfill 3分 \par
\qquad $= P (A) - P (A B) = P (A) - P (A) P (B)$ \dotfill 6分 \par
\qquad $= P (A) (1 - P (B)) = P (A) P (\bar{B})$ \dotfill 9分 \par
所以$A$和$\bar{B}$相互独立。\dotfill 10分
\end{proof}

\vfill

\end{document}
